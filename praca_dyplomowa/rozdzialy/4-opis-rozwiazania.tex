\newpage % Rozdziały zaczynamy od nowej strony.
\section{Opis rozwiązania}
\subsection{Zbiór danych}
Do wstępnej obróbki zbioru danych wykorzystano skrypty dostarczone przez organizatorów konkursu razem z rozwiązaniem \textit{baseline} w niezmienionej postaci. Przetwarzają one adnotacje w formacie GeoJSON na obrazy w formacie TIFF i rozdzielczości identycznej jak odpowiadające im zdjęcia. Dla jednej pary zdjęć (przed i po katastrofie) generowane są maksymalnie cztery jednokanałowe obrazy:
\begin{enumerate}
\item Pikselowa maska budynków o wartościach 0 -- brak budynku, 1 -- budynek.
\item Pikselowa maska dróg o wartościach 0 -- brak drogi, 1 -- droga.
\item Pikselowa maska dróg o wartościach od 0 do 7, odpowiadających różnym ograniczeniom prędkości. Nieużywana w tej pracy.
\item Pikselowa maska budynków i dróg po katastrofie z klasyfikacją zniszczeń. Wartości: 0 -- 
