\newpage % Rozdziały zaczynamy od nowej strony.
\section{Przegląd literaratury}

W pracy \cite{unet} zaproponowano nową architekturę dla zadania semantycznej segmentacji obrazów. 
W 2012 roku ukazała się praca \cite{alexnet}, która miała kluczowe znaczenie dla dalszego rozwoju wizji komputerowej. Zastosowanie głębokiej splotowej sieci neuronowej w połączeniu z kilkoma nowatorskimi technikami (funkcja aktywacji ReLU, dropout, augmentacja danych treningowych) pozwoliło znacząco poprawić wynik klasyfikacji obrazów ze zbioru ImageNet.\\
W \cite{vggnet} ...\\
W pracy \cite{resnet} wprowadzono koncepcję połączenia rezydualnego w głębokiej sieci neuronowej, które pozwoliło efektywnie trenować sieci o znacznie większej liczbie warstw niż do tej pory. Wyuczony przez przez autorów model o nazwie ResNet uzyskał rekordowe wyniki na zbiorze ImageNet i stał się podstawą architektury wielu modeli wizji komputerowej na następne lata.\\

